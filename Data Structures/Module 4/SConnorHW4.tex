%=======================02-713 LaTeX template, following the 15-210 template==================
%
% You don't need to use LaTeX or this template, but you must turn your homework in as
% a typeset PDF somehow.
%
% How to use:
%    1. Update your information in section "A" below
%    2. Write your answers in section "B" below. Precede answers for all 
%       parts of a question with the command "\question{n}{desc}" where n is
%       the question number and "desc" is a short, one-line description of 
%       the problem. There is no need to restate the problem.
%    3. If a question has multiple parts, precede the answer to part x with the
%       command "\part{x}".
%    4. If a problem asks you to design an algorithm, use the commands
%       \algorithm, \correctness, \runtime to precede your discussion of the 
%       description of the algorithm, its correctness, and its running time, respectively.
%    5. You can include graphics by using the command \includegraphics{FILENAME}
%
\documentclass[11pt]{article}
\usepackage{amsmath,amssymb,amsthm}
\usepackage{tikz}
\usetikzlibrary{arrows}
\usepackage{graphicx}
\usepackage[margin=1in]{geometry}
\usepackage{fancyhdr}
\usepackage{mathtools}
\usepackage{placeins}
\usepackage{listings}
\usepackage{color}

\definecolor{dkgreen}{rgb}{0,0.6,0}
\definecolor{gray}{rgb}{0.5,0.5,0.5}
\definecolor{mauve}{rgb}{0.58,0,0.82}

\lstset{frame=none,
  language=Java,
  aboveskip=3mm,
  belowskip=3mm,
  showstringspaces=false,
  columns=flexible,
  basicstyle={\small\ttfamily},
  numbers=none,
  numberstyle=\tiny\color{gray},
  keywordstyle=\color{blue},
  commentstyle=\color{dkgreen},
  stringstyle=\color{mauve},
  breaklines=true,
  breakatwhitespace=true,
  tabsize=3
}

\setlength{\parindent}{0pt}
\setlength{\parskip}{5pt plus 1pt}
\setlength{\headheight}{13.6pt}
\newcommand\question[2]{\vspace{.25in}\hrule\textbf{#1 #2}\vspace{.5em}\hrule\vspace{.10in}}
\renewcommand\part[1]{\vspace{.10in}\textbf{(#1)}}
\newcommand\algorithm{\vspace{.10in}\textbf{Algorithm: }}
\newcommand\correctness{\vspace{.10in}\textbf{Correctness: }}
\newcommand\runtime{\vspace{.10in}\textbf{Running time: }}
\pagestyle{fancyplain}
\lhead{\textbf{\NAME}}
\chead{\textbf{HW\HWNUM}}
\rhead{\today}
\begin{document}\raggedright
%Section A==============Change the values below to match your information==================
\newcommand\NAME{Sean Connor}  % your name
\newcommand\HWNUM{4}              % the homework number
%Section B==============Put your answers to the questions below here=======================

\question{Q1}{}
This method will return the given singly linked list in the reverse order. It acts by rearranging pointers. Head and tail pointers are correctly set.
\begin{lstlisting}
/* Java/pseudocode to reverse the elements of a singly-linked list */
reverseSLL(list)
	prevNode = null
	currNode = list.head
	nextNode = null

	while(currNode != null) // list traversal
		nextNode = currNode.next
		currNode.next = prevNode
		prevNode = currNode
		currNode = nextNode

	list.tail = list.head
	list.head = prevNode

	return list
    
\end{lstlisting}

\question{Q2}{}
For both unordered lists and arrays, one must iterate through each element in the list/array until the value is found. It is assumed that the ‘order’ of the unordered list/array is completely random, in which case the average number of nodes accessed in a search will be n/2.
\newline
\newline
For an ordered array (random access), a binary search can be employed to search for a particular node. The time complexity of binary search is O(log2 n) or simply O(log n).  The best case is Ω(1); however, the average performance of binary search is Θ(log n). Thus, the average number of nodes accessed will be log n.
\newline
\newline
For an ordered list, average number of nodes accessed would depend on the implementation – namely whether list is random or sequential access. From the Java API on binarySearch (https://docs.oracle.com/javase/7/docs/api/java/util/Collections.html)
\begin{quotation}
This method runs in log(n) time for a "random access" list (which provides near-constant-time positional access). If the specified list does not implement the RandomAccess interface and is large, this method will do an iterator-based binary search that performs O(n) link traversals and O(log n) element comparisons.
\end{quotation}
Thus, the average number of nodes for a sequential access list will be n/2, and for a random access list log n. 

\question{Q3}{}
This question is deceptively tricky because there are several different cases to evaluate. 
\begin{lstlisting}
/* Java/pseudocode to interchange the mth and nth elements of a singly-linked list */
swap(list,m,n)
	curr = list.head
	prevM = null
	currM = null
	nextM = null
	prevN = null
	currN = null
	nextN = null

	if(curr == M) // case head --> m
		currM = curr
		nextM = curr.next

	if(curr == N) // case head --> n
		currN = curr
		nextN = curr.next

	while(curr != null) // list traversal
		if(curr.next == M)
			prevM = curr
			currM = curr.next
			nextM = curr.next.next
		if(curr.next == N)
			prevN = curr
			currN = curr.next
			nextN = curr.next.next
		curr = curr.next

	if(prevM == null) // case head --> m
		list.head = currN
		currN.next = nextM

		prevN.next = currM
		currM.next = nextN

	else if(prevM == null) // case head --> n
		list.head = currM
		currM.next = nextN

		prevM.next = currN
		currN.next = nextM
	
	else
		prevM.next = currN // m-1 --> n
		currN.next = nextM // n --> m+1
		
		prevN.next = currM // n-1 --> m
		currM.next = nextN // m --> n+1	
\end{lstlisting}

\question{Q4}{}
\begin{lstlisting}
/* Java/pseudocode to implement the deque as a doubly-linked list (not circular, no header) with InsertLeft and DeleteRight methods */
public class Deque
	private Node head = null // left
	private Node tail = null // right

	private class Node
		Node left
		Node right
		Item data

		Node(Item data, Node left, Node right)
			this.data = data
			this.left = left
			this.right = right

	boolean isEmpty()
		if(head == null AND tail == null)
			return true
		else 
			return false

	void insertLeft(Item data)
		Node node = new Node(data, null, head) // data, left (head), right (tail)
		head = node

	Item deleteRight()
		if(isEmpty())
			throw exception
		else
			Node temp = tail
			tail.left.right = tail.right // set node n-1 next to null
			tail = tail.left // set tail to node n-1
			return temp	
\end{lstlisting}

\question{Q5}{}
\begin{lstlisting}
/* Java/pseudocode to implement the deque as a doubly-linked circular list with a header and InsertRight and DeleteLeft methods. */
public class Deque
	
	private class Node
		Node left // head
		Node right // tail
		Item data

		Node(Item data, Node left, Node right)
			this.data = data
			this.left = left // head
			this.right = right // tail

	Node header = new Node(null,null,null)
	
	boolean isEmpty()
		if(head == null AND tail == null)
			return true
		else 
			return false

	void insertRight(Item data)
		if(isEmpty())
			Node node = new Node(data, node, node) // points to itself
		else
			Node node = new Node(data, header.left, header.right)
			header.right.right = node
			node.right = header.left
			header.right = node

	Item deleteRight()
		if(isEmpty())
			throw exception
		else
			Node temp = header.left // temp = first (head) node
			header.right.right = header.left.right // set trail node next to header.next
			header.left = header.left.right // update where header head points to
			return temp	
\end{lstlisting}

\question{Q6}{}
My answer to this question does require additional working out, but these are the basics.
\begin{lstlisting}
/* Java/pseudocode for hybrid implementation of list capable of handling multiple stacks/queues */
public class HybridArray{

    private int size;
    Node[] data = new Node[size];
    Stack free = new Stack(size);

    public HybridArray(int size){
        this.size = size;
        for(int i = 0; i < size; i++){
            free.push(i);
        }
    }

    // Implement stack as an array
    private class Stack{
        private int arr[];
        private int size;
        private int top = 0;

        private Stack(int size){
            this.size = size;
            arr = new int[size];
        }

        private void push(int element){
            if(top == size) {
                System.out.println("Invalid.");
            }
            else {
                arr[top] = element;
                top++;
            }
        }

        private int pop(){
            if (isEmpty()){
                System.out.println("Invalid.");
            }
            return arr[top-1];
        }

        private boolean isEmpty(){
            if (top == 0) {
                return true;
            }
            return false;
        }
    }

    // Implement node to hold data and pointer
    private class Node{
        int data;
        int next;

        private Node(int data, int next){
            this.data = data;
            this.next = next;
        }
    }


    // Methods to push and pop to/from the free index stack
    public int freeIndex(){
        if(free.isEmpty()){
            System.out.println("Invalid.");
        }
        return free.pop();
    }

    public void returnFree(int index){
        free.push(index);
    }

    // Stack methods
    public void push(int data, int next){
        int freeIndex = free.pop(); // obtain an index from the free index stack
        Node node = new Node(data, 0)
        data[freeIndex] = node;
    }

    public Node pop(){
        if(data.isEmpty()){
            throw exception;
        }
        node = startNode
        while(node.next != 0){ // traverse array to find last node in stack
            prevNode = node;
            node = data[node.next];
        }
        data[prevNode].next = 0; // set pointer of new last item to 0
        returnFree(node); // return the index to the free index stack
        return data[node];
    }

    // Queue methods
    public void add(int data, int next){
        int freeIndex = free.pop(); // obtain an index from the free index stack
        Node node = new Node(data, 0)
        data[freeIndex] = node;
    }

    public void remove(){
        temp = startNode;
        startNode = startNode.next; // set new startNode to the next element in queue
        returnFree(temp); // return the index to the free index stack
        return data[temp];
    }
    
}
\end{lstlisting}

\end{document}









