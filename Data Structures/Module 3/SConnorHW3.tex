%=======================02-713 LaTeX template, following the 15-210 template==================
%
% You don't need to use LaTeX or this template, but you must turn your homework in as
% a typeset PDF somehow.
%
% How to use:
%    1. Update your information in section "A" below
%    2. Write your answers in section "B" below. Precede answers for all 
%       parts of a question with the command "\question{n}{desc}" where n is
%       the question number and "desc" is a short, one-line description of 
%       the problem. There is no need to restate the problem.
%    3. If a question has multiple parts, precede the answer to part x with the
%       command "\part{x}".
%    4. If a problem asks you to design an algorithm, use the commands
%       \algorithm, \correctness, \runtime to precede your discussion of the 
%       description of the algorithm, its correctness, and its running time, respectively.
%    5. You can include graphics by using the command \includegraphics{FILENAME}
%
\documentclass[11pt]{article}
\usepackage{amsmath,amssymb,amsthm}
\usepackage{tikz}
\usetikzlibrary{arrows}
\usepackage{graphicx}
\usepackage[margin=1in]{geometry}
\usepackage{fancyhdr}
\usepackage{mathtools}
\usepackage{placeins}
\usepackage{listings}
\usepackage{color}

\definecolor{dkgreen}{rgb}{0,0.6,0}
\definecolor{gray}{rgb}{0.5,0.5,0.5}
\definecolor{mauve}{rgb}{0.58,0,0.82}

\lstset{frame=none,
  language=Java,
  aboveskip=3mm,
  belowskip=3mm,
  showstringspaces=false,
  columns=flexible,
  basicstyle={\small\ttfamily},
  numbers=none,
  numberstyle=\tiny\color{gray},
  keywordstyle=\color{blue},
  commentstyle=\color{dkgreen},
  stringstyle=\color{mauve},
  breaklines=true,
  breakatwhitespace=true,
  tabsize=3
}

\setlength{\parindent}{0pt}
\setlength{\parskip}{5pt plus 1pt}
\setlength{\headheight}{13.6pt}
\newcommand\question[2]{\vspace{.25in}\hrule\textbf{#1 #2}\vspace{.5em}\hrule\vspace{.10in}}
\renewcommand\part[1]{\vspace{.10in}\textbf{(#1)}}
\newcommand\algorithm{\vspace{.10in}\textbf{Algorithm: }}
\newcommand\correctness{\vspace{.10in}\textbf{Correctness: }}
\newcommand\runtime{\vspace{.10in}\textbf{Running time: }}
\pagestyle{fancyplain}
\lhead{\textbf{\NAME}}
\chead{\textbf{HW\HWNUM}}
\rhead{\today}
\begin{document}\raggedright
%Section A==============Change the values below to match your information==================
\newcommand\NAME{Sean Connor}  % your name
\newcommand\HWNUM{3}              % the homework number
%Section B==============Put your answers to the questions below here=======================

\question{Q1}{}
\begin{lstlisting}
/* Algorithm to recursively compute a+b, where a and b are non-neg integers */
add( int a, int b )
	if( b == 0 )
		return a;
	return add( a+1, b-1 );

/* Test case */
a = 3, b = 2
add( 3, 2 )
	return add( 4, 1 )
		return add( 5, 0 )
			return 5
= 5
\end{lstlisting}

\question{Q2}{}
We need to find the sum of the array by adding each item recursively, and then divide by the number of elements in the array (N = array.length). There will be three cases, as follows:
\begin{enumerate}
	\item i == 0
	\item i == N-1 (base case)
	\item all other values of i
\end{enumerate}
Thus, the Java/pseudocode will be as follows:
\begin{lstlisting}
/* Java/pseudocode for average of array */
averageArray( int[] A, int i, int N )
	if( i == 0 )  // Case 1 - this case returns the average
		return ( A[i] + averageArray( A, i+1, N )  ) /  N;
	else if( i == N-1 )  // Case 2 (base case)
		return A[i];
	else  // Case 3
		return A[i] + averageArray( A, i+1, N );
\end{lstlisting}
Then one could simply call averageArray(A, 0, A.length) with the initial call having i = 0 as a requirement for proper calculation of average.

\question{Q3}{}
There are three cases here as follows:
\begin{enumerate}
	\item Case 1 (base case 1) $\rightarrow$ n == 0, return f0
	\item Case 2 (base case 2) $\rightarrow$ n == 1, return f1
	\item Case 3  $\rightarrow$ n $>$ 1, return gfib(f0, f1, n-1) + gfib(f0, f1, n-2)
\end{enumerate}
The Java/pseudocode is as follows:
\begin{lstlisting}
/* Java/pseudocode for generalized fibonacci */
gfib( int f0, int f1, int n )
	if( n == 0 )
		return f0;
	else if( n == 1 )
		return f1;
	else
		return gfib(f0, f1, n-1) + gfib(f0, f1, n-2);
\end{lstlisting}

\question{Q4}{}
For this question I wrote and executed a Java program to compute Ackerman(2,2). The code is as follows:
\begin{lstlisting}
public class Ackerman
{
    public static void main(String[] args)
    {
        System.out.println("Ackerman(2,2) = " + ack(2,2));
    }

    public static int ack(int m, int n)
    {
        if( m == 0 )
        {
            return n + 1;
        }
        else if( n == 0 )
        {
            return ack(m-1, 1);
        }
        else
        {
            return ack(m-1, ack(m, n-1));
        }
    }
}
\end{lstlisting}
The result of this is:
\begin{lstlisting}
C:\Users\Sean\Documents\jhu-cs\Data Structures\Module 3>java Ackerman
Ackerman(2,2) = 7
\end{lstlisting}

\question{Q5}{}
For this problem, it is simple to convert the recursive function to an iterative function using a while loop. The Java/pseudocode for this is as follows:
\begin{lstlisting}
int rec(int n)
	bool x = f(n);
	int a = n;
	while( x == FALSE )
		/* any group of statements that do not change the value of n */
		a = g(a);
		x = f(a);
	return 0;
\end{lstlisting}
In this way, f(n) is initially calculated using n. By setting a variable `a' equal to n, n remains unchanged and a can be updated to equal g(n) and so on recursively. Finally, when x = f(a) returns TRUE, the method returns 0.

\question{Q6}{}
There are four obvious ways to implement the queue. They are...
\begin{itemize}
	\item Sorted array
	\item Sorted linked list
	\item Unsorted array
	\item Unsorted linked list
\end{itemize}
For my ADT, I chose to use an unsorted array, with methods modified to account for the additional requirements (namely searching for highest priority item in array and shifting items for deletion).
\begin{lstlisting}
ADT PriorityQueue
	Data
		An empty unsorted array of values with a reference to the first (front) item, which is highest priority. The queue stores information on two parts of an item - the data and the priority
	Methods
		isEmpty
			Input  	None
			Precondition  	None
			Process  	Check if the queue contains any data items
			Postcondition  	None
			Output  	Return true if queue is empty, and false otherwise
		Insert
			Input   	A data item to be stored in the queue
			Precondition  	Item has an assigned priority
			Process  	Store an item at the rear of a queue
			Postcondition  	The queue contains one additional data item
			Output  	None
		Delete
			Input  	None
			Precondition  	Queue contains meaningful data values
			Process  	Remove the highest priority element, accomplished by searching the queue
			Postcondition 	 Shift all elements greater than index of removed element left by one element
			Output  	Return deleted value	
		Peek
			Input  	None
			Precondition	  Queue contains meaningful data values
			Process 	 Search queue for highest priority item
			Postcondition 	 None
			Output 	Return the value of the data item at the front of the queue (highest priority)
end ADT PriorityQueue	
\end{lstlisting}

\end{document}









